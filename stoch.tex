\chapter{Stochastic parameterization schemes}
\label{stoch}

Physical parameterization schemes represent averages over unresolved,
subgrid-scale processes and those averages necessarily exhibit
fluctuations depending on the precise realization of subgrid-scale
fields.  Hence physical parameterization schemes should include a random,
``stochastic'' component.  To represent this random contribution to
forecast error, ensemble prediction systems now use routinely stochastic
parameterization schemes.
Stochastic parameterization schemes have been shown to improve the
probabilistic skill of weather forecasts on short and medium forecast
timescales 
\citep[e.g.][]{Be09,Be11,leutbecher2017stochastic}.
Less known is the fact that stochastic parameterization schemes have also the
potential to reduce systematic model error 
\citep[e.g.][]{berner2017stochastic}.

The stochastic parameterization suite in WRF comprises a number of
stochastic parameterization schemes, targeted at representing different aspects 
of uncertainty (see Table \ref{stoch_table}). Generally, one or more random
perturbation fields are generated and used to perturb tendencies or
parameters in the physical parameterization schemes.  It is of importance
that the random perturbation fields are characterized by spatial and temporal
correlations, which can be prescribed.

The benefits of stochastic parameterization schemes are most evident in ensemble 
prediction systems, where the added ensemble diversity leads to a more reliable 
ensemble spread and improved probabilistic forecast skill \citep[e.g.][]{Be15}.

%--------------------------------------------------------------------------------------------
\begin{table}[h!]
\begin{center}
\caption{Stochastic parameterization suite}
\begin{tabular}{ | l | l | l |}
\hline
Scheme          & Name & Perturbations to \\ 
\hline
SPPT  & Stochastically perturbed    &  u-,v-,$\theta$- and $q_v$-tendencies\\
      & parameterization tendencies &  from PBL and convection\\
SKEBS & Stochastic kinetic-energy   &  full ($=$ physics + dynamics) \\
      & backscatter scheme          &  $u_{\rm rot}$-,$v_{\rm rot}$- and $\theta$- tendencies\\ 
SPP & Stochastically perturbed      &  select parameters from\\
    & parameter scheme              &  PBL and convection \\
RPF & Random perturbation           &  interface must be \\
    & field                         &  provided by user\\
\hline
\end{tabular}
\end{center}
\label{stoch_table}
\end{table}

%--------------------------------------------------------------------------------------------
\section {Stochastically perturbed physics tendencies (SPPT)}
%--------------------------------------------------------------------------------------------
The Stochastically Perturbed Parameterization Tendency scheme
(SPPT) is based on the assumption that there is uncertainty
in the parameterized physics tendencies and that this 
uncertainty is proportional to the net physics tendency \citep{Bu99,Pa09}.
Consequently, SPPT perturbs the net physical tendency of temperature, 
humidity and wind components at each time step with a multiplicative 
random coefficient:
\begin{equation}
   \frac{\partial X}{\partial t}= D + (1+ r) \sum_i P_i \,.
\end{equation}
Here, ${\partial X}/{\partial t}$ denotes the total tendency in variable $X$, 
$D$ is the tendency from the dynamical core, 
$P_i$ the tendency from the i-th physics scheme 
and $r$ is a two-dimensional, Gaussian distributed zero-mean random
perturbation field with spatial and temporal correlations. 
Depending on
the implementation, the SPPT scheme can use up to three patterns with
different spatial and time scales and a vertical tapering function which 
tapers perturbations near the surface and for the highest model levels.
The WRF implementation uses by default only a single perturbation field \citep{Be15}.
and perturbs the tendencies from the PBL and convection schemes, but not radiation
or micro-physics.

%--------------------------------------------------------------------------------------------
\section{Stochastic kinetic-energy backscatter scheme (SKEBS)}
%--------------------------------------------------------------------------------------------
The Stochastic Kinetic Energy Backscatter scheme (SKEB) aims to represent
the interactions of turbulent eddies near or below the
truncation level with the resolved state. Some of the eddy-eddy interactions will
will dissipate, but others lead to a resolved-scale effect.
Originally developed for large eddy simulation applications, these ideas were 
adapted for numerical weather prediction by
\citet{Sh05} and \citet{Be09}.
SKEBS generates streamfunction perturbations with spatio-temporal correlations. These
can be weighted with the dissipation rate from numerical diffusion, convection and orographic 
wave drag. 

In the WRF implementation, no dissipation weighting is used and the 
perturbations can be extended to the potential temperature field \citep{Be11}.
Wind perturbations are proportional to the square
root of the kinetic-energy backscatter rate, and temperature
perturbations are proportional to the potential energy backscatter rate.

A comparison of SKEBS and SPPT shows that SKEBS introduces most model diversity in the 
free atmosphere and for dynamical variables, 
while SPPT is most active in regions with large tendencies, e.g., in areas with convection 
and near the surface \citep{Be15}.
%--------------------------------------------------------------------------------------------
\section {Stochastically perturbed parameterization scheme (SPP)}
%--------------------------------------------------------------------------------------------
To describe the statistics over the over unresolved, subgrid-scale processes, 
physical parameterization schemes have a number of deterministic parameters.
While some of them have no uncertainty, e.g. the gravitation constant, 
others have a large uncertainties due to 
measurement errors and heterogeneities in the underlying physical process
e.g. the processes represented by the parameter ``roughness length'' in the PBL scheme or
the shape parameter characterizing the particle size distribution in the micro-physics 
scheme.  

The stochastically perturbed parameterization scheme (SPP)
perturbs the values of uncertain key parameters and provides 
thus a way to represent uncertainties within a particular parameterization scheme
\citep{Bo08,Ha11a}.
Perturbations to e.g. the vertical mass flux profile 
in the convection scheme or the shape parameter of the particle size distribution in the
micro-physics scheme allow in principle to represent structural model error.
Since the in the statistical sciences fundamental distinction between ``parameter error'' and 
``structural model error '' is blurred when parameters in the physical parameterization schemes 
are perturbed, the scheme is called ``Stochastically perturbed {/it parameterization} scheme''
rather than ``Stochastically perturbed {/it parameter} scheme''. 

The SPP released in WRF is designed in close collaborations with the physical
parameterization developers and under active development. Hence it is only
available for selected physics packages, namely the Grell Freitas convection
scheme and the the MYNN boundary layer scheme \citep{jankov2017performance}.

The member diversity introduced by SPP tends to be smaller than that of SKEBS and SPPT 
and is in itself typically not enough to generate reliable ensemble spread 
\citep{Be15}.
SPP can be augmented by one of the other model-error schemes or used in itself to study 
forecast sensitivity to a particular parameter setting.

%--------------------------------------------------------------------------------------------
\section {Random perturbation field (RPT)}
%--------------------------------------------------------------------------------------------
WRF enables the user to generate a 3-D Gaussian random perturbation field
with prescribed spatial and temporal correlations and use to it perturb
parameters of interest.  Since the interface has to be provided by the
user, this option is not recommended for novice users.

%--------------------------------------------------------------------------------------------
\section {Stochastic Perturbations to the lateral boundary conditions}
%--------------------------------------------------------------------------------------------
The stochastic tendencies in WRF are typically treated as physics-tendencies and change the 
perturbed fields at each time step within the WRF domain. However, since the stochastic
perturbation field is also generated on the boundary, it can be used to perturb the lateral boundary 
boxes. We refer to the user guide for details.

%--------------------------------------------------------------------------------------------
\section {Stochastic Perturbations to the lateral boundary conditions}
%--------------------------------------------------------------------------------------------
\begin{figure}[h!]
  \includegraphics[trim=0cm 15cm 0 10cm,scale=.80]{figures/stoch_pattern.png}
    \caption{ [Will be replaced with better figure]
    Perturbation patterns for three different length scales: a) convection-permitting scale, 
    b) meso-scale, c) synoptic scale.  
    }
  \label{fig_stoch_pattern}
\end{figure}
