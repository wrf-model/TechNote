\chapter{Introduction}
\label{introduction_chap}
The Weather Research and Forecasting (WRF) Model is
an atmospheric modeling system designed for both research and numerical 
weather prediction.  WRF is an open-source community model, and it has been 
adopted for research at universities and governmental laboratories, 
for operational forecasting by governmental and private entities,
and for commercial applications by industry.  WRF development began in the latter 
half of the 1990's with the goals being to build a system shared by research 
and operations and to create a next-generation numerical weather 
prediction (NWP) capability.  The new modeling system has become a 
common platform on which the broad research community 
can develop capabilities that can transition to operations, 
while the extra scrutiny of performance in operations
could guide and accelerate development.  The WRF system was developed
through a partnership of the National 
Center for Atmospheric Research (NCAR), the National Oceanic and Atmospheric 
Administration (NOAA) (represented by the National Centers for Environmental 
Prediction (NCEP) and the NOAA Earth System Research Laboratory (ESRL)), 
the United States Air Force, the Naval Research Laboratory, the 
University of Oklahoma, and the Federal Aviation Administration.  
For more information 
on the history of the WRF model development see \citet{powers17}.

\section {Advanced Research WRF (ARW)}

ARW is a configuration of the WRF system featuring
the ARW dynamics solver together with other compatible components 
to produce a simulation.  Thus, 
it is a subset of the WRF system that, in addition to the specific solver, 
encompasses physics schemes, numerics/dynamics options, 
initialization routines, and a data assimilation package (WRFDA).  
ARW consists of flexible, modular, portable code that is 
efficient in computing environments ranging from laptops to 
massively-parallel supercomputers and is readily-configurable for 
a variety of applications.  Its extensive menu of options 
for physical process schemes and for configuring numerics  
reflect a history of community input
and make it a powerful NWP tool.  WRFDA offers a variety of data assimilation approaches and that can ingest a 
broad array of observation types.  In addition, for earth system prediction needs 
beyond basic weather forecasting, ARW supports a number of tailored capabilities, including 
WRF-Chem (atmospheric chemistry), 
WRF-Hydro (hydrological modeling), and WRF-Fire (wildland fire modeling).

ARW is supported as a community model, facilitating system development
and broad use for research, operations, and education.  It supports  
atmospheric simulations across scales from large-eddy to global.  
ARW's applications include real-time NWP, weather events and 
atmospheric-process studies, data assimilation development, 
parameterized-physics development, regional climate simulation, air 
quality modeling, atmosphere-ocean coupling, and idealized-atmosphere studies.  

Figure 1.1 depicts the principal components of the ARW system. 
The WRF Software Framework (WSF) is the infrastructure 
that contains the dynamics solver, physics packages, utilities
for initialization, WRFDA, and integrated capabilities such as WRF-Chem, WRF-Hydro, WRF-Fire, etc.  
The WSF also contains the NMM-E dynamics solver, which is used by NCEP in the operational HWRF model.
The Mesoscale and Microscale Meteorology Laboratory of NCAR provides
support for the ARW, and oversees the WRF repository and releases.

%
% Figure 1.1
%
\begin{figure}
  \centering
  \includegraphics[width=6.5in]{figures/component.pdf}
  \caption{\label{figure:1}Advanced Research WRF system components.}
\end{figure}

This technical note focuses on the scientific and algorithmic 
approaches in the ARW Version 4, including its dynamical solver, physics options,
initialization capabilities, boundary conditions, and grid-nesting techniques.  
The WSF provides the software infrastructure, although this infrastructure is not 
reviewed in this technical note. 
Additionally, while WRF-Chem, WRF-Hydro, WRF-Fire, and other tailored systems 
use the ARW solver, they are also 
not covered by this technical note.  For information on 
actually running the ARW system, the ARW Version 4 Modeling System User's Guide
covers model operation
 
{\noindent(http://www2.mmm.ucar.edu/wrf/users/docs/user\_guide\_V4/WRFUsersGuide.pdf).  }

The following section highlights the major features of the 
ARW Version 4, first released in May 2018.

\section {Major Features of the ARW System, Version 4}

\vskip 12pt
{\noindent\bf ARW Dynamics Solver}
\vskip 12pt

\begin{description}
\setlength{\itemsep}{-5pt}
\item{$\bullet$} {\em Equations:}
Fully-compressible, Eulerian nonhydrostatic equations solver with 
a run-time hydrostatic option available. Conserves dry air mass and scalar mass.
%
\item{$\bullet$} {\em Prognostic Variables:}
Velocity components $u$ and $v$ in Cartesian coordinate, vertical velocity $w$, 
perturbation moist potential temperature, perturbation geopotential, 
and perturbation dry-air surface pressure.
Optionally, turbulent kinetic energy and any number of scalars
such as water vapor mixing ratio, rain/snow mixing ratio,
cloud water/ice mixing ratio, and chemical species and tracers.
%
\item{$\bullet$} {\em Vertical Coordinate:}
Terrain-following, mass-based, hybrid sigma-pressure vertical coordinate based on dry hydrostatic presure, 
with vertical grid stretching permitted.
Top of the model is a constant pressure surface.
%
\item{$\bullet$} {\em Horizontal Grid:}
Arakawa C-grid staggering. 
%
\item{$\bullet$} {\em Time Integration:}
Time-split integration using a 2nd- or 3rd-order Runge-Kutta scheme with
smaller time step for acoustic and gravity-wave modes. 
Variable time step capability.
%
\item{$\bullet$} {\em Spatial Discretization:}
2nd- to 6th-order advection options in horizontal and vertical.
%
\item{$\bullet$} {\em Turbulent Mixing and Model Filters:} Sub-grid scale
turbulence formulation in both coordinate and physical space.
Divergence damping, external-mode filtering, vertically implicit
acoustic step off-centering. Explicit filter option.
%
\item{$\bullet$} {\em Initial Conditions:}
Three dimensional for real-data, and one-, two- and 
three-dimensional for idealized data. 
Digital filtering initialization (DFI) capability 
available (real-data cases).
%
\item{$\bullet$} {\em Lateral Boundary Conditions:} 
Periodic, open, symmetric, and specified options available.
%
\item{$\bullet$} {\em Top Boundary Conditions:} 
Gravity wave absorbing (diffusion, Rayleigh damping, or implicit 
Rayleigh damping for vertical velocity).  
Constant pressure level at top boundary along a material surface. 
Rigid lid option.
%
\item{$\bullet$} {\em Bottom Boundary Conditions:} 
Frictional or free-slip.
%
\item{$\bullet$} {\em Earth's Rotation:}
Full Coriolis terms included.
%
\item{$\bullet$} {\em Mapping to Sphere:} 
Four map projections are supported for real-data simulation: 
polar stereographic, Lambert conformal, Mercator, and 
latitude-longitude (allowing rotated pole). 
Curvature terms included.
%
\item{$\bullet$} {\em Nesting:} 
One-way interactive, two-way interactive, and moving nests.
Multiple levels and integer ratios.
%
\item{$\bullet$} {\em Nudging:}
Grid, spectral, and observation nudging capabilities. 
%
\item{$\bullet$} {\em Global Grid:}
Global simulation capability using polar Fourier filter and 
periodic east-west conditions. 
%
\item{$\bullet$} {\em Tropical Channel:}
Tropical channel capability using periodic east-west and specified north-south 
lateral boundary conditions.
\end{description}

%\newpage
\vskip 12pt
{\noindent\bf Model Physics}
\vskip 12pt

\begin{description}
\setlength{\itemsep}{-5pt}
\item{$\bullet$} {\em Microphysics:} Schemes ranging from simplified
physics suitable for idealized studies to mixed-phase, multi-moment, bin, and aerosol-aware
approaches to support process studies and accurate NWP.
%
\item{$\bullet$} {\em Cumulus parameterizations:}
Deep and shallow convection, adjustment, mass-flux, and scale-aware schemes available.
%
\item{$\bullet$} {\em Surface physics:}
Multi-layer land surface models ranging from a simple thermal model to full
vegetation and soil moisture models, including snow cover and sea ice.  Urban parameterizations are available.
%
\item{$\bullet$} {\em Planetary boundary layer physics:}
Turbulent kinetic energy prediction or non-local $K$ schemes.
%
\item{$\bullet$} {\em Atmospheric radiation physics:} 
Longwave and shortwave schemes with multiple spectral bands and a 
simple shortwave scheme suitable for climate and weather applications.  
Cloud effects and surface fluxes are included.
\end{description}

\vskip 12pt
{\noindent\bf WRFDA System}
\vskip 12pt

\begin{description}
\setlength{\itemsep}{-5pt}
\item{$\bullet$} Data assimilation capability merged into WRF software framework.
%
\item{$\bullet$} Incremental formulation of the model-space cost function.
%
\item{$\bullet$} Quasi-Newton or conjugate gradient minimization algorithms.
%
\item{$\bullet$} Analysis increments on unstaggered Arakawa-A grid.
%
\item{$\bullet$} Representation of the horizontal component of background 
error ${\bf B}$ via recursive filters (regional) or power spectra (global). The
vertical component is applied through projection onto climatologically-averaged 
eigenvectors of vertical error. Horizontal/vertical errors are
non-separable (horizontal scales vary with vertical eigenvector).
%
\item{$\bullet$}  Background cost function ($J_b$) preconditioning 
via a control variable transform ${\rm U}$ defined as ${\bf B}={\rm U} {\rm U}^T$.
%
\item{$\bullet$} Flexible choice of background error model and control variables.
%
\item{$\bullet$} Background error covariances estimated via either the
NMC-method of averaged forecast differences or suitably-averaged
ensemble perturbations.
%
\item{$\bullet$} 3DVAR, 4DVAR, hybrid ensemble-3DVAR approaches available. 
Global and regional, multi-model DA capability.
%
\end{description}


\vskip 12pt
{\noindent\bf WRF Software Framework}
\vskip 12pt

\begin{description}
\setlength{\itemsep}{-5pt}
\item{$\bullet$} Highly modular, single-source code for maintainability.
%
\item{$\bullet$} Two-level domain decomposition for distributed- and 
shared-memory parallel computation.
%
\item{$\bullet$} Portable across a range of available computing platforms.
%
\item{$\bullet$} Support for multiple physics modules.
%
\item{$\bullet$}
Separation of scientific codes from parallelization and other 
architecture-specific aspects.
%
\item{$\bullet$}
Input/output Application Program Interface (API) enabling various external
packages to be installed with WRF, thus allowing WRF
to easily support various data formats.
%
\item{$\bullet$}
Efficient execution on a range of computing platforms
(distributed and shared memory, vector
and scalar types).
%
%\item{$\bullet$}
%Use of Earth System Modeling Framework (ESMF) and interoperable as an ESMF
%component.
%
%\item{$\bullet$}
%Model coupling API enabling WRF to be coupled with other models such as
%ocean, and land models using ESMF, MCT, or MCEL.
\end{description}

\vskip 12pt
{\noindent\bf WRF-Chem, WRF-Hydro and WRF-Fire}
\vskip 12pt

{\em WRF-Chem} is a full online atmospheric chemistry model with many options and some
interactions with the physics via aerosols affecting radiation and microphysics when appropriate
options are chosen. Typically it requires emission source maps as additional inputs.  Further information on WRF-Chem can be found at
https://ruc.noaa.gov/wrf/wrf-chem/

{\em WRF-Hydro} is a surface hydrological model that can be run online with WRF or offline,
interacting with WRF through the land-surface model. This model can calculate the land water
budget terms including streamflow given additional data for routing. It operates on a higher-resolution
sub-grid relative to the atmospheric model.  Further information on WRF-Hydro can be found on their web page at
https://ral.ucar.edu/projects/wrf\_hydro/overview

{\em WRF-Fire} operates as a coupled wildland fire model that keeps track of a fire front at the
sub-grid level. It interacts with the model via winds and heat fluxes and requires fuel information as
an additional input.  Further information concerning WRF-Fire can be found in the ARW Users Guide 
http://www2.mmm.ucar.edu/wrf/users/docs/user\_guide\_v4/contents.html

Additionally, a brief description of these components, along with references to further information, can by found in \citet{powers17}.